\documentclass[9pt,twocolumn]{article}

\usepackage[margin=.5in]{geometry}
\usepackage{sectsty}
\usepackage{graphicx}
\usepackage{caption}
\usepackage{listings}
\usepackage[none]{hyphenat}

\title{
  \textbf{CMSC 128: Introduction to Software Engineering} \\
  First Examination Reviewer
}
\date{}
\sectionfont{\centering\fontsize{12}{14}\selectfont}
\subsectionfont{\fontsize{10}{10}\selectfont}
\captionsetup{justification=centering}
\pagenumbering{gobble}
\lstset{
  basicstyle={\ttfamily},
  breaklines=true,
  showstringspaces=false,
  breakatwhitespace=true
}

\begin{document}
\maketitle

\section*{Introduction}
  \subsection*{Program}
  A set of instructions written to perform a certain task. May or may not have an "interface". \textbf{Example:} Parse the contents of a text file and format its contents appropriately.

  \subsection*{Application}
  A software intended to run on a specific environment and perform a task.

  \subsection*{Software}
  \begin{itemize}
    \item Composed of many programs
    \item Can perform a number of tasks
    \item Documented
  \end{itemize}

\section*{Software Engineering}
  Software engineering is a term that was introduced by \textbf{Friedrich Bauer} in October 1968 NATO Software Engineering conference in Garmisch, Germany. It is the \emph{creative activity of understanding the business problem, coming up wih an idea for solution, and desining the 'blueprints' of the solution}.

  Software engineering is the aplication of the disciplined approach for the development and maintenance of computer software. A software engineer must understand the customer's business needs and design software to help meet them. Each customer is unique.

  Software engineers must possess:
  \begin{enumerate}
    \item The ability to quickly learn new and diverse disciplines and business processes.
    \item The ability to communicate with domain experts, extract an abstract model of the problem, and formulate a solution that makes sense in the context of customer's business.
    \item The ability to design a software system that will realize the proposed solution and gracefully evolve with the evolving business needs for many years in the future.
  \end{enumerate}

\section*{History of Software Engineering}
  \begin{description}
    \item[John Tukey] first coined software engineering in 1958.
    \item[Alan Turing] first established the theoretical concept of a computer in the 1930s.
    \item[Muhammad Al-Khawarezmi] is a 9th century mathematician who introduced the \textbf{concept of algorithm}.
    \item[Ada Lovelace] is the first computer programmer who made an algorithm concrete when she first programmed it.
    \item[Alan Kay] pioneered work on window-based graphical user interface. He is also known for coining the term \textbf{object-oriented programming} in 1966.
    \item[Ali Mili] contributed to the \textbf{formalization of software fault tolerance}, now a major concern for developing secure software systems.
    \item[Barry Boehm] contributed to the area of software engineering economics and software metrics. He has also introduced the \textbf{spiral model} for software development.
    \item[Bill Joy] contributed to the development of the \textbf{Unix operating system} and the \textbf{Java} programming language.
    \item[Brian Kernighan and Dennis Ritchie] were instrumetal in developing the \textbf{Unix operating system} and the \textbf{C programming language}.
    \item[C.A.R. Hoare] introduced the concepts of \textbf{assertions and program proof of correctness}, designed and analyzed well-known algorithms, and developed communicating sequential processes, a formal language for the specification of concurrent processes.
    \item[David Parnas] introduced the concepts of \textbf{information hiding}, software interfaces, and software modularity. He has also contributed to software engineering education and the \textbf{ethical responsibilities of a software engineer}.
    \item[Donald Knuth] is known fo the design of many well-known computer algorithms and the use of rigorous mathematical techniques for the formal analysis of the complexity of algorithms.
    \item[Edsger Dijkstra] introduced the concept of \textbf{structured programming} and carried out in-depth studies of the problems of concurrency and synchronization needed in complex distributed systems.
    \item[Gamma 4] with Erich Gamma, Richard Helm, Ralph Johnsonn, and John Vlissides introduced the concept of \textbf{object-oriented design patters} to facilitate software design reuse.
    \item[Fred Brooks] contributed to the development of the \textbf{OS/360 operating system software}. He is also known for introducing mythical man-month in software project management.
    \item[Friedrich Bauer] introduced the use of \textbf{stacks} for expression evaluation in programming language compilers, contributed to the development of the \textbf{ALGOL} and coined the term \emph{software engineering} in the NATO conference held in Germany in 1968.
    \item[Grace Hopper] contributed extensively to the \textbf{first compiler} and the \textbf{COBOL programming language} in the 1950s.
    \item[Grady Booch] is known for his method for \textbf{object-oriented analysis} and design and his co-development of the \textbf{Unified Modeling Language (UML)}.
    \item[John Backus] is well known for the invention of \textbf{FORTRAN}, \textbf{compiler optimization}, and the \textbf{Backus-Naur Form} for the formal description of programming language syntax.
    \item[Michael Fagan] contibuted extensively to the area of \textbf{software inspection}.
    \item[Niklaus Wirth] introduced the \textbf{Pascal programming language} and other languages, and contributed to the idea of \textbf{decomposition} and \textbf{stepwise refinement}.
    \item[Ole-Johan Dahl and Kristen Nygaard] introduced the \textbf{Simula language} which was the first object-oriented programming language.
    \item[Peter Naur] is known for his contributions to the creation of the \textbf{ALGOL programming language} and the introduction of formal syntax description language.
    \item[Tom DeMarco and Edward Yourdon] introduced the structured analysis and design approach for specifying and designing software systems.
    \item[Watts Humphrey] is known for his work on the \textbf{capability maturity model} developed by the software engineering institute and the personal software process.
    \item[Winston Royce] contributed the well-known model for the management of large software systems whih was later named the \textbf{Waterfall model}.
  \end{description}

\section*{Software Development Team}
  \subsection*{Project Manager (PM)}
  \begin{itemize}
    \item Setup administrative direction for the project
    \item Lays out the initial scheduling and project goals
    \item Listens to feedback from end users and Business Analyst
    \item Setup project meetings and resources
    \item Determines project scope
    \item Responsible for documents such as SRS
  \end{itemize}

  \subsection*{Team Leader (TL)}
  \begin{itemize}
    \item Leads the developer in creating a module, software or application.
    \item Coordinates in other team leaders and with the project manager.
    \item Key point personnel in the managing the pacing of the team in terms of development.
    \item Suggests and/or decides on the implementation to be used as well as the choice of technology.
  \end{itemize}

  \subsection*{Business Analyst}
  \begin{itemize}
    \item Examines the existing or ideal organiation and design of systems including businesses, departments, and organizations.
    \item May conduct a gap analysis
    \item Gives advices on areas that might be optimized or improved
    \item Begins to work out a project strategy
  \end{itemize}

  \subsection*{Developer}
  \begin{itemize}
    \item Creates and implements a set of progras needed for the software
    \item Programs are created based on the agreed architecture and design by the team.
    \item Creates strategies in the implementation of solutions at different levels of system abstraction and technology used.
    \item Also known as \emph{Software Developer}, \emph{Programmer}, etc.
  \end{itemize}

  \subsection*{Designer}
  \begin{itemize}
    \item Responsible for the look and feel of the application based on the team feedback
    \item Create mockups for review by the team
  \end{itemize}

  \subsection*{Tester}
  \begin{itemize}
    \item Ensures that the application/software behaves properly on a combination of input and scenarios.
    \item Provides test cases to mimic realistic usage of the software.
    \item Provides comments for improvements
  \end{itemize}

  Sometimes, you would hear some of these terms:
  \begin{description}
    \item[Full Stack Developer] is knowledgeable in various levels of implementation (i.e. business logic, sever, network, interface).
    \item[UX Designer] is responsible for enhancing user experience and satisfaction through improved usability, aesthetics, accessibility, and performace of a piece of software.
    \item[Subject Matter Expert (SME)] is a person with in-depth knowledge from both a business and IT perspective that when shared with others, significantly enhances performance within the organization.
  \end{description}

\end{document}